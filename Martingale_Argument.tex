\section*{Martingale Argument for Concentration}
The martingale argument is due to~\cite{lovett2012}. Consider $X_1(t), \ldots, X_n(t)$, $n$ simple random walks (SRWs) on $\mathbb{Z}$. There are $s$ stationary points and $m$ moving points at any instant; $m+s=n$. The top $s$ are frozen. We're interested in the maximum process $W(t)$:
\begin{equation}
  W(t) := \sup_{k \leq t} \sup_{i \leq n} \{ |X_i(k)| \}
  \label{eq:max-process-defn}
\end{equation}

Let the filtration $\mathcal{F}_t$ be ``information about all the points up to time $t$. Then $X_i(t)$ is a martingale with respect to the filtration for each $i$ because,
\begin{equation}
  X_i(t) = X_i(t-1) + \left\{ 
    \begin{array}{cc} 
      0 & \textrm{if frozen} \\
      \pm 1 & \textrm{with equal probability if moving}  
    \end{array} \right.
\end{equation}

Let $T$ be the final time in the martingale, and Doob's inequality gives us,
\begin{equation}
  \Prob(\sup_{k \leq T} |X_i(k)| \geq a) \leq \int_{\sup_{k \leq T} |X_i(k)| \geq a} X_T d\Prob
  \label{eq:doob-for-martingale}
\end{equation}
However, the $X_i$ are bounded by $1$ in absolute value, and Azuma's inequality gives us 
\begin{equation}
  \Prob(\sup_{k \leq T} |X_i(k)| \geq u\sqrt{T}) \leq 2 e^{-u^2/2}.
  \label{eq:azuma-improving-doob}
\end{equation}
which is great. Now, define the random variable $Y$ as follows: 
\begin{equation}
  Y := \# \{i | \sup_{k \leq T} |X_i(k)| \geq u\sqrt{T} \}
\end{equation}
and its expectation is
\begin{equation}
  E[Y] = n 2 e^{-u^2/2}
\end{equation} using~\Eqref{eq:azuma-improving-doob} and linearity over indicators. The following set inclusion is easy
\begin{equation}
  \{ W(T) \geq u\sqrt{T} + 1 \} \subset \{ Y \geq s+1 \}
\end{equation}
since one needs at least one point to be able to move when $s+1$ are at $u\sqrt{T}$ to be able to get to $u\sqrt{T}+1$. Then,
\begin{equation}
  \Prob (W(T) \geq u\sqrt{T} + 1) \leq \frac{n}{s+1} \exp(-u^2/2)
  \label{eq:concentration-for-w}
\end{equation}
gives us concentration for fixed $n,s$. This tells us that we ought to keep $n/s$ constant if we're to send $n \to \infty$.
